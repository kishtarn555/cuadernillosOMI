Definimos una inversión como una parejas \((i,j)\) en un arreglo tal que \(1\leq i < j \leq N\) y también \(A_i > A_j\). Es decir, una pareja de números que estén al revés de como deberían estarlo en un orden de menor a mayor.

Dado un arreglo de \(N\) enteros, imprime cuantas inversiones hay en él.

\subsubsection*{Ejemplo}
\begin{casebox3}
	\ecase{
		4\\
		3 2 6 1
	}
	{4}
	{
		Las inversiones son: \\
		El 3 con el 2, \\
		el 3 con el 1,\\
		el 2 con el 1, \\
		y el 6 con el 1. 
	}	
\end{casebox3}
\subsubsection*{Límites}
\begin{plimits}
	\item \(1\leq N \leq 5000\)
	\item \(1\leq A_i \leq 10^9\)
\end{plimits}