La revista “algofashion” dijo esta semana que los números a la moda son aquellos que pueden ser representados como la suma de dos números pertenecientes a la secuencia de Fibonacci.

Recordemos que la secuencia de Fibonacci empieza con dos 1. Y luego cada número será resultado de la suma de los dos anteriores.
\[F_1=F_2=1\]
\[F_n=F_{n-1}+F_{n-2}\]

De forma que los primeros números de la secuencia son:
\[1,1,2,3,5,8,13,21\ldots\]

Karel acaba de leer la revista y ahora quiere responder \(T\) preguntas, cada pregunta será del tipo: ¿El número \(x_i\) está de moda?

Como amigo, debes hacer un código que responda las dudas de Karel.

\subsubsection*{Entrada}
En la primera línea vendrá el valor de \(T\), cuantas preguntas tiene Karel.

En las siguientes \(T\) líneas vendrán las preguntas de Karel, una por línea. Cada pregunta consiste en un solo entero \(x_i\).

\subsubsection*{Salida}
Imprime \(T\) líneas, cada una siendo la respuesta a una pregunta de Karel. La línea \(i\) debe ser “SI” si \(x_i\) está a la moda y debe ser “NO” si no está a la moda. 

\subsubsection*{Caso ejemplo}	

\begin{casebox2}
	\scase{
		3\\
		5\\
		6\\
		10
	}
	{
		SI\\
		NO\\
		SI
	}
\end{casebox2}

\subsubsection*{Límites}

\begin{itemize}
	\setlength{\parskip}{1pt}	
	\item \(1\leq T \leq 100\)
	\item \(1\leq x_i \leq 10^{18}\)
\end{itemize}