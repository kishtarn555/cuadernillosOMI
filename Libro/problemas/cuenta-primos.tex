Dado \(L\) y \(R\), encuentra cuantos números primos\footnote{Un número es primo si y solo si tiene exactamente dos divisores positivos, el \(1\) y el mismo.} hay entre \(L\) y \(R\), incluyendo \(L\) y \(R\).


\subsubsection*{Entrada}
Dos enteros \(L\) y \(R\).
\subsubsection*{Salida}
La cantidad de números primos en el rango.
\subsubsection{Caso ejemplo}
\begin{casebox3}
	\ecase{1 7}{4}{Los primos contados son: \(2,3,5\) y \(7\)}	
\end{casebox3}

\subsubsection*{Subtareas}
\begin{plimits}
	\item (50 pts) \(1\leq L\leq R\leq 10^3\)
	\item (50 pts) \(1\leq L\leq R\leq 10^5\)
\end{plimits}

\omegalink{Cuenta-primos}