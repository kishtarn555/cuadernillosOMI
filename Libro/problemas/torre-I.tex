Estas en una torre la cual tiene \(N\) pisos.

El piso \(i\) tendrá \(K_i\) pasadizos que le permiten ir a pisos superiores. El único piso sin pasadizos es el último, el piso \(N\).

Tu quieres ver cuantas formas hay de ir del piso \(1\) al \(N\). Dos maneras de ir del piso \(1\) al \(N\) son diferentes si en una usas un pasadizo y en la otra no.

Nota: esta prohibido tomar los pasadizos de un piso superior a uno inferior

\subsubsection*{Entrada}
Dos enteros \(N\), la cantidad de pisos en la torre.

En las siguientes \(N-1\) lineas vendrá la descripción de cada piso. 

Cada piso constará de un entero \(K_i\) representando cuantos pasadizos hay allí que conectan hacia arriba. Seguido, vendrán \(K_i\) enteros diferentes, representando a que pisos tienes un pasadizo. Solo podrás ir a pisos superiores.

\subsubsection*{Salida}
Imprime un entero que represente la cantidad de formas de ir del piso \(1\) al piso \(N\).

\subsubsection*{Ejemplo}
\begin{casebox3}
	\ecase{
		4\\
		2 2 3\\
		2 3 4\\
		1 4  
	}{3}
	{
		Tienes tres formas de llegar:\\
		\(1\rightarrow2\rightarrow3\rightarrow4\)\\
		\(1\rightarrow2\rightarrow4\)\\
		\(1\rightarrow3\rightarrow4\)\\
	}
\end{casebox3}

\subsubsection*{Límites}
\begin{plimits}
	\item \(1\leq N\leq 16\)
	\item \(1\leq K_i< N\)
\end{plimits}