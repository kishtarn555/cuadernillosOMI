Polycarpio acaba de inventar un nuevo juego. El juego consiste en un tablero de \(n\) botones, cada botón tiene un valor entero positivo, cero o negativo. El tablero tendrá dos enteros \(a\) y \(b\). (\(a\) representa el inicio y \(b\) la meta).

El objetivo es llegar de \(a\) a \(b\) en la menor de pasos. Para jugar debes elegir un botón del tablero y sumarle a \(a\) el valor del botón. La cantidad de veces que puedes presionar un botón es ilimitada y el orden en el que lo presionas puede ser cualquiera.

Siempre es posible llegar de \(a\) a \(b\).

\subsubsection*{Entrada}
En la primera línea recibirás \(a\) y \(b\) (el inicio y la meta).

En la segunda línea un numero ¿(n) que representa el número de botones.

\(n\) líneas con el valor de cada botón \(n_i\).

\subsubsection*{Salida}	
\begin{casebox3}
	\ecase{
		1 2\\
		3\\
		2\\
		1\\
		-1
	}{1}{Presiona el segundo botón con valor \(1\).}
	\ecase{
		1 7\\
		5\\
		20\\
		30\\
		-10\\
		1\\
		-2
	}{4}{
		En este caso los botones a presionar\\ 
		son -2, -2, -10, 20.		
	}
\end{casebox3}
\subsubsection*{Límites}
\begin{plimits}
	\item \(-10^5 \leq a,b \leq 10^5\)
	\item \(n \leq 5\)
\end{plimits}

Para 10\%:  \(n = 2\)

Fuente: OMIAGS 2020

\omegalink{Juego-de-los-botones}
