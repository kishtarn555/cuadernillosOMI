\chapter[Búsqueda Exhaustiva]{Búsqueda exhaustiva\\ Fuerza bruta}
Muchas veces en nuestra vida hemos tenido que buscar algo, una foto en nuestra galería del teléfono, una palabra en el diccionario, una carta dentro de un mazo, etc. Y probablemente, con la experiencia, hemos aprendido algunas intuiciones sobre como buscar cosas, en esta parte trabajaremos un poco más en desarrollar esta intuición e ideas.

En la olimpiada de informática también debemos buscar cosas. Ya sea encontrar la solución al problema o solo utilizar una búsqueda como paso intermedio, ser buenos haciendo búsquedas nos abrirá la puerta a muchos problemas y técnicas.

Para comenzar con entendiendo las búsquedas comenzaremos con la búsqueda más sencilla, la búsqueda exhaustiva, también conocida como fuerza bruta. 

Lo primero que se requiere para poder buscar es definir el espacio de búsqueda, ¿dónde podría estar lo que queremos encontrar? De la respuesta dependerá como haremos la búsqueda. Por ejemplo, si buscamos la posición de un valor en un arreglo, el espacio de búsqueda es el arreglo.

Una vez que definamos donde podría estar la respuesta, lo que hacemos con la exhaustiva es explorar absolutamente todo el espacio de búsqueda, todos los candidatos que podrían ser lo que buscamos hasta dar con la respuesta. Por esto se le conoce como fuerza bruta, porque aprovecha el poder computacional para procesar todo hasta dar con la respuesta.

Se te enseñara las búsquedas exhaustivas más comunes, así como una forma general. En concreto, los espacios de búsquedas que usaremos son:

\begin{plimits}
	\item Espacio lineal
	\item Parejas de elementos
	\item Ordenes (Permutaciones)
	\item Subcojuntos
\end{plimits}
 
 Y terminaremos este capitulo aprendiendo como hacer búsqueda exhaustiva de forma general usando recursión e iteración.
 

 


