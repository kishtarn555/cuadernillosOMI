\section*{Problemas de práctica}
\addcontentsline{toc}{section}{Problemas de práctica}
\markright{Problemas de práctica}


\begin{exercise}
	\problema[problemas/multiplos-de-cinco]{TODO Multiplos de cinco}{\omegauplink{multiplos-cinco}}
\end{exercise}

\begin{exercise}
	\problema[problemas/divisores]{TODO: Divisores del entero}{\omegauplink{divisores-entero}}
\end{exercise}

\begin{exercise}
	\problema[problemas/suma-de-digitos]{A la suma de digitos}{\omegauplink{m-suma-digitos}}
\end{exercise}

\begin{exercise}
	\problema[problemas/escalera-de-fer]{Escalera de fer}{TODO}
\end{exercise}

\begin{exercise}
	\problema[problemas/bicicleta-de-karel]{Bicicleta de Karel I}{\omegauplink{bicicleta-de-karel-i}}
\end{exercise}

\begin{exercise}
	\problema[problemas/contar-capicuas]{Contar capicúas}{\omegauplink{Contar-capicuas}}
\end{exercise}

\begin{exercise}
	\problema[problemas/cuenta-primos]{Cuenta primos}{\omegauplink{Cuenta-primos}}
\end{exercise}

\begin{exercise}
	\problema[contar-sumas-pares]{TODO Puntuación de la ferretería}{\omegauplink{contar-sumas-pares}}
\end{exercise}

\begin{exercise}
	\problema[algofashion-fibo]{TODO Números de moda}{\omegauplink{moda-fibonacci}}
\end{exercise}

\begin{exercise}
	\problema[terreno-valioso]{TODO Terreno más valios}{\omegauplink{TODO terreno-valioso}}
\end{exercise}

\begin{exercise}
	\problema[suma-modulo-k]{TODO Suma modular}{\omegauplink{TODO suma-modulo-k}}
\end{exercise}

\begin{exercise}
	\problema[warel-roba-diamantes]{Warel roba diamantes}{\omegauplink{warel-roba-diamantes}}
\end{exercise}

\begin{exercise}
	\problema[imprime-binarios]{Imprime números binarios}{\omegauplink{Imprime-binario}}
\end{exercise}

\begin{exercise}
	\problema[reverse-matrix]{Invirtiendo una matriz}{\omegauplink{TODO reverse-matrix}}
\end{exercise}

\begin{exercise}
	Cuenta cuantas sumas diferentes distintas hay en los subconjuntos de un arreglo.
	
	\problema{Contando LaserTag justos}{TODO}
\end{exercise}

\begin{exercise}
	\problema[torre-I]{Subiendo la torre}{TODO}
\end{exercise}

\begin{exercise}
	\problema{Mapas}{\omegauplink{OMI-2020-Mapas}}
\end{exercise}

\begin{exercise}
	TODO FIX THIS:
	Dado un arreglo de \(N\) elementos, responde \(Q\) queries. \(N\leq 20\) \(Q \leq 10^5\)
	
	Cada query serán dos elementos \(s\) y \(t\), determina si existe un subconjunto de tamaño \(t\) de suma \(s\). \(s \leq 1000\)
	
	\problema{Preguntas sumisticas}{TODO}
\end{exercise}