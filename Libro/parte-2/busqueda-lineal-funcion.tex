\section{Búsqueda lineal con función de validación}
\markright{BÚSQUEDA LINEAL CON FUNCIÓN DE VALIDACIÓN}

Hasta ahorita hemos visto problemas donde revisar si un candidato era la respuesta o no bastaba con un simple condicional, pero este no siempre es el caso.

Varías veces, para revisar si un valor es solución a nuestro problema, vamos a tener que necesitar un poco más de código e ideas. Veamos un problema de este estilo.

\subsection{Ejemplo 1.3}
\addcontentsline{toc}{section}{Ejemplo 1.3}

\subsection*{Descripción}
Karel tiene \(N\) listones de distintas longitudes enteras. Karel quiere hacer pulseras con ellos, por lo que tomará cada uno de los listones y los cortará para que las pulseras usen segmentos del mismo tamaño.

A Karel le gustan los enteros, entonces la longitud de los segmentos también ha de serla. Además, Karel no quiere que sobre listón sin usar ¿Cuántos diferentes tamaños de segmento se pueden elegir?

\subsection*{Entrada}
Un entero \(N\), indicando la cantidad de listones
En la siguiente línea, \(N\) enteros indicando las longitudes de los listones. Llamemos \(A[i]\) a la longitud del listón \(i\).

\subsection*{Salida}
La cantidad de opciones para el tamaño de los segmentos

\subsection*{Caso ejemplo}
\begin{casebox3}
	\ecase{
		5 \\ 
		10 30 20}
	{3}
	{Las longitudes pueden ser 1, 2 y 5}
\end{casebox3}	
\subsection*{Límites}
\begin{plimits}
	\item \(1\leq N \leq 100\)
	\item \(1\leq A[i] \leq 5000\)
\end{plimits}

\subsection*{Código}
Encontremos con búsqueda lineal todos los tamaños de segmento que cumplen y contemos cuantos son.

Primero veamos que los tamaños de listón deben estar entre 1 y 5000. Más concreto, entre \(1\) y \(min (A[1],A[2],\cdots ,A[N])\). Esto es porque el tamaño del segmento debe ser entero, debe ser por lo menos 1 (ya que un segmento de tamaño 0 o menor no tiene sentido para este problema) y no puede ser más largo que el listón más corto.

Ya hemos visto como se ve una búsqueda lineal y la que usaremos en este caso sería:

\begin{lstlisting}
	cin >> N;
	for (int i=0; i< N; i++) {
		cin >> A[i];
	}
	
	int minA=A[0];
	for (int i=1; i < N; i++) {
		minA=min(minA, A[i]); /* encuentra el liston mas
		pequeno. */
	}
	respuesta = 0;
	for (int s =1; s <= minA; s++) {
		if (es s es un tamano de segmento valido) {
			respuesta++;
		}
	}
	cout << respuesta
	
\end{lstlisting}

Pero el reto ahora es el chequeo de “es s es un segmento de tamaño valido”.

Para esto necesitamos un poco más de trabajo. Veamos un solo listón. Si queremos cortarlo en segmentos de tamaño s sin que sobre, ¿qué tiene que cumplir s con relación al listón? Así es, que es, que s divida a la longitud del listón. Y podemos ver que s tiene que cumplir esto para todos los listones

Entonces, para ver que s sea una opción válida, hay que ver que s divida a todos los enteros en la lista de listones.

Para lograr esto, creemos una función booleana que se encargue de validar s.
\begin{lstlisting}
	bool validar (int s) {
		bool respuesta = true;
		for (int i=0; i< N; i++) {
			if (A[i]%s!=0){
				respuesta = false;
				break;
			}
		}
		return respuesta;
	}
\end{lstlisting}

Entonces con esta función obtenemos que el código de la búsqueda lineal ahora es:
\begin{lstlisting}
	respuesta = 0;
	for (int s =1; s <= minA; s++) {
		if (validar(s)) {
			respuesta++;
		}
	}
	cout << respuesta
	
\end{lstlisting}

Y con esto logramos completar el problema.

\subsection{Complejidad}
La búsqueda lineal la hacemos sobre el valor de A, pero, además, por cada iteración de la búsqueda lineal, hacemos un ciclo que revisa la condición para que s sea contada.

Entonces, la complejidad nos queda como:   \(O(Busqueda\times Validar)=O(AN)\).

Como \(A\leq 5000\) y \(N\leq 100\). Nos queda que \(AN\leq 5\times 10^5\), lo cual corre en menos de un segundo.


\section*{Problemas de práctica}
\addcontentsline{toc}{section}{Problemas de práctica}


\begin{exercise}
	\problema[problemas/bicicleta-de-karel]{Bicicleta de Karel I}{\omegauplink{bicicleta-de-karel-i}}
\end{exercise}

\begin{exercise}
	\problema[problemas/contar-capicuas]{Contar capicúas}{\omegauplink{Contar-capicuas}}
\end{exercise}

\begin{exercise}
	\problema[problemas/cuenta-primos]{Cuenta primos}{\omegauplink{Cuenta-primos}}
\end{exercise}
