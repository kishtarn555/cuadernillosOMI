\section{Búsqueda en los reales}
\markright{BÚSQUEDA EN LOS REALES}
Hay veces en la que no estamos buscando un valor entero, si no en vez querremos encontrar un valor real a cierta precisión.

El problema dirá algo por el estilo de: imprime la respuesta con precisión de \(10^{-6}\), esto quiere decir que tu respuesta no debe estar alejada de la solución por más de \(10^{-6}\).

Para estos problemas debes cambiar la condición de paro de \(a==b\), por \(b-a < precision\). Aunque en realidad, recomiendo poner una precisión un poco más ajustada ya que no tiende a aumentar mucho el costo y permite una programación más relajada.

De forma que ahora la binaria se verá como:

\begin{lstlisting}
double a =0, b=1e9;
double epsilon = 1e-8;
while(b-a>= epsilon) {
	double m = (a+b)/2;
	if (funciona(m)) {
		b=m;
	} else {
		a=m;
	}
}
\end{lstlisting}

\section{Ejemplo: Raíz cuadrada real}
Calcula la raíz cuadrada de un número A con precisión absoluta de \(10^{-4}\).

\textbf{Ejemplo}\\
\begin{casebox2}
	\scase{10}{3.1622}
	\scase{16}{4.0000}
\end{casebox2}

\textbf{Límites}
\begin{plimits}
	\item \(1\leq A \leq 10^9\)
\end{plimits}

\subsection*{Solución}
Usamos las mismas observaciones que usamos para el ejemplo ``De área a perímetro", pero esta vez lo haremos con \lstinline|double| y nos detenemos basados en una precisión.
\begin{lstlisting}
	double raiz(double A) {
		double a=0, b=A;
		while (b-a >= 1e-5) {
			double m=(a+b)/2;
			if (m*m < A) {
				a=m;
			} else {
				b=m;
			}
		}
		return a;
	}
\end{lstlisting}

\subsection{Complejidad}
La complejidad de este procedimiento sigue siendo \(O(log(Espacio))\), pero ahora debemos encontrar el tamaño del espacio de búsqueda. 

Con un poco de matemáticas veremos que es \(E=\frac{R-L}{precision}\), en el caso de ejemplo la precisión es pues: \(\frac{A}{10^{-4}}\), aunque nosotros usamos una precisión de \(10^{-5}\), por lo que es \(\frac{A}{10^{-5}}\).

Por lo tanto, la complejidad del ejemplo anterior es:
\[O(logE)\]
\[ = O(log\frac{A}{10^{-5}})\]
\[ = O(log(A)-log({10^{-5}}))\]
\[ = O(log(A)+log({10^{5}}))\]

Entonces, poner una precisión de \(10^-5\) solo nos aumenta un costo de \(log(10^5)=16.6\), o en otras palabras, 17 preguntas extra.

\section*{Problemas de práctica}
\addcontentsline{toc}{section}{Problemas de práctica}

\begin{exercise}
	\problema{Carreras de la Formula \(\pi\) }{TODO}
\end{exercise}

\begin{exercise}
	\problema{Planetas}{\omegauplink{planetas}}
\end{exercise}

\begin{exercise}
\problema{Carretera}{\omegauplink{carretera}}
\end{exercise}