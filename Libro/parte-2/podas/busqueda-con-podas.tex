\chapter[Búsqueda con podas]{Búsqueda con podas\\ Backtracking}

Sigamos con una búsqueda que evita probar absolutamente todo el espacio de búsqueda, en este caso es la búsqueda de vuelta a atrás, o backtracking en inglés.

Esta búsqueda es similar a la búsqueda exhaustiva, pero estando atentos a no explorar candidatos en los que nos demos cuenta que no encontraremos lo que buscamos.

Como acostumbramos veamos un ejemplo.

\section*{Ejemplo: Subconjunto de suma K}

Dada una lista de \(N\) enteros positivos, determina si existe un subconjunto que sus elementos sumen \(K\).

\textbf{Entrada}\\
Dos enteros \(N\) y \(K\) --- La cantidad de enteros y la suma que queremos.

En la segunda línea hay \(N\) enteros \(A_1, A_2, \ldots A_N\) --- Los valores de la lista.
\textbf{Salida}\\
Imprime \verb|SI| en caso de que exista un subconjunto de suma \(K\), o imprime \verb|NO| en caso contrario.

\textbf{Ejemplos}\\
\begin{casebox3}
	\ecase{
	4 12\\
	5 8 3 1 7	
	} {
	SI
	}
	{
		\(5+7=12\)
	}
	\ecase{
		6 11\\
		10 5 2 3 5 7
	} {
		NO
	}
	{
	}
	\ecase{
	4 22\\
	5 5 5 11 
	} {
	NO
	}
	{	
	}
\end{casebox3}

\textbf{Límites}
\begin{plimits}
	\item \(1\leq N,K \leq 200\)
	\item \(1\leq A_i \leq 200\)
\end{plimits}
	
\textbf{Subtareas}
\begin{plimits}
	\item (30 pts) \(1\leq N \leq 20\)
	\item (30 pts)
	\begin{plimits}
		\item \(1\leq N \leq 30\)
		\item \(1\leq K \leq 9\)		
	\end{plimits}
	\item (40 pts) Sin restricciones adicionales
\end{plimits}

\subsection*{Solución}
Vamos a ir construyendo una solución desde la fuerza bruta hasta obtener la de 100 puntos.

\subsubsection*{Fuerza bruta}
La fuerza bruta es crear todos los subconjuntos posibles, viendo si la suma es lo que queremos o no, recordemos que en código se ve:

\begin{minipage}{\linewidth}
\begin{lstlisting}
void buscar(int posicion, int suma) {
	if (posicion==N) {
		if (suma==K) {
			cout << "SI";
			exit(0);
		}
		return;
	}
	//prueba a incluir arr[pos]
	buscar(posicion+1, suma+arr[posicion]);
	
	//prueba a excluir arr[pos]
	buscar(posicion+1, suma);
}
\end{lstlisting}
\end{minipage}

Esta búsqueda nos produce el siguiente diagrama de recursión en el caso ejemplo uno. Cada llamada recursiva es representada por dos números, el primero es posición y el segundo es suma.


\Tree [.buscar(0,0) 
	[.1\\5 
		[.2\\13 
			[.3\\16
				[.4\\17 
					5\\24 
					5\\17
				]
				[.4\\16
					5\\23
					5\\16
				]
			]
			[.3\\13
				[.4\\14
					5\\21
					5\\14
				]
				[.4\\13
					5\\20
					5\\13
				]
			]
		]
		[.2\\5
			[.3\\8
				[.4\\9
					5\\16
					5\\9				
				]
				[.4\\8
				5\\15
				5\\8				
				]
			]			
			[.3\\5
				[.4\\6
					5\\13
					5\\6
				]
				[.4\\5
					\textbf{5}\\\textbf{12}
				]
			]		
		]
	]	
]




En total, se requirió de 31 llamadas recursivas y revisamos 15 subconjuntos, ¿pero realmente fue necesario hacer todo lo que hicimos?

\subsection*{Primera poda}
Pongamos atención en la llamada de \lstinline|buscar(2, 13)|, desde aquí hacemos 14 llamadas recursivas, que si lo pensamos ya son innecesarias. Si la suma ya supero 12, que es lo que buscamos y no hay números negativos, entonces no importa que hagamos, jamas encontraremos 12.

Por lo tanto, la primera forma de mejorar la fuerza bruta es detener la búsqueda si superamos \(K\). Esto se ve en código:

\begin{minipage}{\linewidth}
	\begin{lstlisting}
		void buscar(int posicion, int suma) {
			if (suma > K) {
				return;
			}
			if (posicion==N) {
				if (suma==K) {
					cout << "SI";
					exit(0);
				}
				return;
			}
			//prueba a incluir arr[pos]
			buscar(posicion+1, suma+arr[posicion]);
			
			//prueba a excluir arr[pos]
			buscar(posicion+1, suma);
		}
	\end{lstlisting}
\end{minipage}

De forma que ahora el diagrama de la recursión es mucho menor.

\Tree [.buscar(0,0) 
[.1\\5 
		[.2\\13 		
		]
		[.2\\5
			[.3\\8
				[.4\\9
					5\\16
					5\\9				
				]
				[.4\\8
					5\\15
					5\\8				
				]
			]
			[.3\\5
				[.4\\6
					5\\13
					5\\6
				]
				[.4\\5
					\textbf{5}\\\textbf{12}
				]
			]			
		]
	]	
]

Esta poda nos permite sacar la segunda subtarea, esto es porque como \(K\) vale a lo más \(9\), la cantidad de subconjuntos por explorar son aquellos con \(9\) elementos o menos, los cuales para \(n=30\) hay \(22,964,087\), esto lo podemos comprobar con una búsqueda que cuente la cantidad de subconjuntos de menos de \(9\) elementos, o con fórmulas matemáticas que veremos más adelante.

A esta forma de evitar visitar candidatos le llamamos poda, porque estamos podando ramas del diagrama y reduciendo el volumen de búsqueda. Igual que podar un árbol reduce la cantidad de hojas y el volumen de un árbol.

\subsection*{Segunda poda}
Esta no es la única forma de podar, también podemos cortar la búsqueda en con información de lo que ya hemos explorado.

Para ver esto, veamos el tercer caso ejemplo.


\Tree[.0\\0
	[.1\\5
		[.2\\10
			[.3\\15
				4\\26
				4\\15
			]
			[.3\\10
				4\\21
				4\\10			
			]
		]	
		[.\textbf{2}\\\textbf{5}
			[.\textbf{3}\\\textbf{10}
				\textbf{4}\\\textbf{21}
				\textbf{4}\\\textbf{10}			
			]
			[.\textbf{3}\\\textbf{5}
				\textbf{4}\\\textbf{16}
				\textbf{4}\\\textbf{5}
			]				
		]
	]
	[.1\\0
		[.\textbf{2}\\\textbf{5}
			[.\textbf{3}\\\textbf{10}
				\textbf{4}\\\textbf{21}
				\textbf{4}\\\textbf{10}
			]
			[.\textbf{3}\\\textbf{5}
				\textbf{4}\\\textbf{16}
				\textbf{4}\\\textbf{5}
			]
		]
		[.2\\0
			[.3\\5
				4\\16
				4\\5
			]
			[.3\\0
				4\\11
				4\\0
			]
		]
	]	
]

Este diagrama no es afectado por la poda anterior, pero vamos a introducir una nueva poda. Para esto, probablemente te fijaste en que en el diagrama anterior, resaltamos dos ramas del diagrama en negritas.

El motivo por el cual están calcadas es porque son iguales, inician con buscar(2, 5) para llamar a buscar(3, 10) y buscar(3, 5), etc. Y como son iguales, entonces la segunda vez que lo exploremos, ¿podremos ahora encontrar la respuesta? Claro que no, vamos a ver lo que ya habíamos revisado antes.

Por esto, la segunda poda será que en caso de llegar a un buscar con un valor repetido de posicion  y suma, detener la búsqueda.

En código se ve:

\begin{minipage}{\linewidth}
	\begin{lstlisting}
bool marcas[250][250];
void buscar(int posicion, int suma) {
	if (suma > K) 
		return;	
	if (marcas[posicion][suma]) 
		return;
	marcas[posicion][suma]=true;//Marcamos esta pareja de (pos, suma) como ya explorado para no repetir.
	if (posicion==N) {
		if (suma==K) {
			cout << "SI";
			exit(0);
		}
		return;
	}
	//prueba a incluir arr[pos]
	buscar(posicion+1, suma+arr[posicion]);
	
	//prueba a excluir arr[pos]
	buscar(posicion+1, suma);
}
	\end{lstlisting}
\end{minipage}

Y esto produce el siguiente diagrama, donde marcamos con negritas las ramas podadas:

\Tree[.0\\0
[.1\\5
[.2\\10
[.3\\15
4\\26
4\\15
]
[.3\\10
4\\21
4\\10			
]
]	
[.2\\5
[.\textbf{3}\\\textbf{10}		
]
[.3\\5
4\\16
4\\5
]				
]
]
[.1\\0
[.\textbf{2}\\\textbf{5}
]
[.2\\0
[.\textbf{3}\\\textbf{5}
]
[.3\\0
4\\11
4\\0
]
]
]	
]

Esta poda nos pasa de 31 llamadas recursivas y 16 subconjuntos revisados, a solo usar 21 llamadas y revisar 8 subconjuntos.

Esta poda en concreto nos reduce la complejidad a \(O(NK)\), esto es porque vamos a explorar solo cuando los argumentos aparecen por primera vez. Los valores de posición son de \(0\) a \(N\) y los la suma tiene rango de \(0\) a \(K\). Lo cual implica que en total tenemos \((N+1)(K+1)\) posibles pares de valores. 

Hay llamadas repetidas para los pares, pero estas no hacen nada ya que son podadas de forma inmediata, además, estas solo pueden ser llamadas por un buscar no podado.

Complejidad de \(O(NK)\) corre en tiempo sin problemas para los 100 puntos.

Veamos otro ejemplo.

\section*{Ejemplo: Explorando una matriz}

Un explorador tiene que decidir donde construir un camino para llegar de un punto \(A\) a un punto \(B\). 

Para auxiliarse el explorador ha hecho un mapa con los obstáculos que existen. Cuadriculó su mapa y quiere un camino que pase por el menor número de cuadros. El camino sólo puede ir de un cuadro a otro si tienen un lado en común, es decir, no puede avanzar en diagonal, y no puede pasar por un cuadro que contenga un obstáculo. 

Cada cuadro del mapa se identifica por sus coordenadas, primero la fila y después la columna. Las filas están numerados de arriba hacia abajo iniciando con el \(0\). Las columnas están numeradas de izquierda a derecha iniciando con el \(0\).

Escribe un programa que dado un mapa con obstáculos encuentre el menor número de cuadros por los que debe pasar un camino que vaya del punto \(A\) al punto \(B\), incluyendo a los cuadros que contienen a \(A\) y a \(B\).

\textbf{Entrada}\\
En la primera línea los enteros \(N\) y \(M\) el número de filas y columnas del mapa. En cada una de las siguientes.

En las siguientes \(N\) líneas hay \(M\) enteros que pueden ser \verb|0| ó \verb|1|. Es \verb|0| si no hay obstáculo en el cuadro correspondiente y \verb|1| si lo hay.

En la siguiente línea (la penúltima) la fila y columna del punto \(A\).

En la última línea la fila y columna del punto \(B\).

\textbf{Salida}\\
En la primera linea el número de cuadros por los que pasa un camino mínimo entre \(A\) y \(B\)

\textbf{Ejemplo}\\
\begin{casebox2}
	\scase{
	4 5\\
	0 1 0 0 0\\
	0 0 1 1 0\\
	0 1 0 0 0\\
	0 0 0 0 0\\
	3 1\\
	0 2	
	}{9}
\end{casebox2}

\textbf{Límites}
\begin{plimits}
	\item \(1 \leq N, M\leq 50\)
\end{plimits}

Fuente: OMI 1996, Modificado subtask A

\omegauplink{OIEG2013SSC}

\subsection*{Solución}
Comencemos con la fuerza bruta, básicamente, buscaremos la secuencia de pasos más corta para llegar desde el punto \(A\) hasta el punto \(B\).

Para esto, una observación clave es que ningún camino que sea el más corto pasa dos veces por la misma casilla, esto es porque es innecesario caminar en círculos.

Esto implica que la distancia a lo más usa todas las casillas, esto es \(50\times50=2500\).

Ahora, la fuerza bruta la haremos con una función recursiva que vaya probando todas las secuencias de movimientos posibles, llevando cuenta de las coordenadas de donde va y cuantos pasos ha dado:

\begin{minipage}{\linewidth}
\begin{lstlisting}
int N, M;
int respuesta=1e9;
int destino_i, destino_j;
int marcas[55][55];
void explorar(int pos_i, int pos_j, int pasos) {
	if (pos_i < 0 || pos_i >=N)
		return;//Estas fuera del mapa
	if (pos_j < 0 || pos_j >=M)
		return;//Estas fuera del mapa
	if (pos_i==destino_i && pos_j == destino_j) {
		//Comparar respuesta con la que encontramos
		if (respuesta > pasos)
			respuesta=pasos;
		return;
	}
	if (mapa[pos_i][pos_j])return;
	if (marcas[pos_i][pos_j]) return;
	marcas[pos_i][pos_j]=true; //Marcar para no pasar por aqui otra vez.	
	explorar(pos_i-1, pos_j, pasos+1);//arriba
	explorar(pos_i+1, pos_j, pasos+1);//abajo
	explorar(pos_i, pos_j-1, pasos+1);//izquierda
	explorar(pos_i, pos_j+1, pasos+1);//derecha	
	marcas[pos_i][pos_j]=false; //Desmarcar para permitir pasar por aqui en otros caminos.	
}
int main() {
	cin >> N >>M;
	[...]
	int pi, pj;
	cin >> pi >> pj;
	cin >> destino_i>>destino_j;
	explorar(pi, pj, 1);
	cout << respuesta;
}

\end{lstlisting}
\end{minipage}

Esta fuerza bruta prueba todos los caminos posibles del punto \(A\) al punto \(B\), pero es excesivamente lenta y obtenemos 0 puntos, necesitamos mejorarla.

La primera poda que podemos hacer es que si en algun momento llevamos más pasos que la respuesta encontrada hasta ahora, nos detengamos.

\begin{lstlisting}
	if (pasos >=respuesta) return;
\end{lstlisting}

Con esta idea mejoramos a 25 puntos, pero todavía podemos mejorar.

Para esto, veamos que si en algun momento llegamos a la casilla \((i, j)\) con 20 pasos y luego llegamos con 50, de aquí no podremos encontrar un camino más rápido al punto \(B\), pues todos los caminos desde aquí ya fueron explorados y de forma mejor.

Entonces, añadiremos para cada casilla un marcador de con que tiempo llegamos y solo exploraremos una casilla si llegamos a ella más rápido que antes.
\begin{lstlisting}
if (tiempo[pos_i][pos_j]<=pasos) return;
tiempo[pos_i][pos_j]=pasos;
\end{lstlisting}

Esta segunda poda nos pasa la complejidad a: \(O(distanciaMaxima\times N \times M)\) y como ya vimos, distanciaMaxima es \(N\times M\), por lo tanto la complejidad es \(O((NM)^2)\), lo cual equivale a \(50^4=6,250,000\) que corre en tiempo.

Esta complejidad es resultado de que cada casilla es explorada a lo más la distancia máxima de veces.
\pagebreak
\section{Código general}

Entonces, un código de backtracking se va a ver con la siguiente forma:
\begin{lstlisting}
void buscar(solucion) {
	if (se poda solucion)
		return;
	if (solucion es valida) {
		registrar solucion;
	}
	buscar(soluciones siguientes)
}
\end{lstlisting}

\section*{Problemas de práctica}

\begin{exercise}
	\problema{Suma de dígitos}{\omegauplink{Suma-de-digitos}}	
\end{exercise}

\begin{exercise}
	\problema{8 Reinas}{\omegauplink{8Reinas}}
\end{exercise}

\begin{exercise}
TODO: Dado K, R. Determina cuantos números hay menores o iguales que K con R bits  prendidos. Se garantiza que la respuesta es menor que \(10^5\).
\end{exercise}