\chapter*{Búsqueda completa}
\addcontentsline{toc}{chapter}{Búsqueda completa}
\markboth{Búsqueda completa}{Búsqueda completa}

La búsqueda completa, también llamada fuerza bruta, es una técnica donde revisamos todos los posibles candidatos donde podría estar el o los valores que buscamos.

Esta búsqueda completa tiende a ser lenta, muchas veces incluso tiene complejidad exponencial y por esto, tiende a no ser una solución para los 100 puntos. Sin embargo, es muy útil conocerla ya que la fuerza bruta es fácil de pensar y siempre encuentra la respuesta si esta existe.

Ademas, la fuerza bruta suele resolvernos una o dos subtarea, dando un puntaje parcial. Esto es una forma de garantizar puntos en problema donde no se nos ocurra ideas mejores, es nuestra e también hay varios problemas que consisten en empezar de la fuerza bruta e ir mejorando la solución hasta que sea suficientemente buena.

También llega a ser útil para encontrar patrones y condiciones en las respuestas que no nos percatemos viendo la redacción, o para encontrar errores en un código que hayas hecho.

Probablemente esto te suene a búsqueda lineal, y esto es resultado de que esta es un tipo de búsqueda completa en la que revisamos de forma secuencial un rango donde se puede encontrar la respuesta. Pero búsqueda completa incluye más tipos de iteraciones que solo revisar los valores de un ciclo.

Las búsquedas completas suelen buscar la respuesta en:

\begin{plimits}
	\item Los elementos de un arreglo (búsqueda lineal)
	\item Los valores de un rango (búsqueda lineal)
	\item Las parejas
	\item Los subconjuntos
	\item Los ordenes (permutaciones)
\end{plimits}

Para esto, a continuación analizaremos cada una de estos casos y veamos como trabajar con ellos.