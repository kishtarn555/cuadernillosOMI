\chapter*{Búsqueda completa}
\addcontentsline{toc}{chapter}{Búsqueda completa}
\markboth{Búsqueda completa}{Búsqueda completa}

La búsqueda completa, también llamada fuerza bruta, es una técnica donde revisamos todos los posibles candidatos donde podría estar el o los valores que buscamos.

Esta búsqueda completa tiende a ser lenta, muchas veces incluso tiene complejidad exponencial y por esto, suele no ser la solución para los 100 puntos. Sin embargo, es muy útil conocerla ya que la fuerza bruta es fácil de pensar y siempre encuentra la respuesta si esta existe.

Ademas, la fuerza bruta suele resolvernos una o dos subtareas, dando un puntaje parcial. Esto es una forma de garantizar puntos en problema donde no se nos ocurra ideas mejores-

También hay muchos problemas que consisten en empezar de la fuerza bruta e ir mejorando de allí hasta que sea suficientemente buena.

No solo eso, además llega a ser útil para encontrar patrones y condiciones en las respuestas que no nos percatemos viendo la redacción, o para encontrar errores en un código que hayas hecho.


Las búsquedas completas suelen buscar la respuesta en:

\begin{plimits}
	\item Los elementos de un arreglo (búsqueda lineal)
	\item Los valores de un rango (búsqueda lineal)
	\item Las parejas
	\item Los ordenes o permutaciones
	\item Los subconjuntos
	\item Las cadenas de decisiones
\end{plimits}

Probablemente reconozcas la búsqueda lineal, y esto es resultado de que esta es la búsqueda completa más sencilla en la que revisamos de forma secuencial un rango donde se puede encontrar la respuesta. Pero búsqueda completa incluye más tipos de iteraciones que solo revisar los valores de un ciclo.

Veamos a continuación cada una de estos casos y aprendamos a trabajar con ellos.